\chapter{Conclusions and Sustainability of ADRs} 

    \section{Summary of Findings}
        This thesis aimed to explore the role and effectiveness of Architectural Design Records (ADRs) in documenting architectural decisions within open source software repositories. It also examined the architectural decision process in software engineering and how ADRs fit into it's lifecycle. By performing numerous topic modelling techniques such as TF-IDF, LDA, and BERTopic, we analyzed and enriched a dataset of ADRs extracted from GitHub.
        With the help of LLMs, findings revealed a diverse range of topics within the documents, with an emphasis in areas such as frontend development, data management, user authentication, and cloud infrastructure management that concern architects and engineers alike, in line with the current industry trends. We were also able to identify popular tactics, topics and patterns from the general software architecture landscape such as microservices and containerization, application tenancy and API design that satisfy common quality attributes of systems. More niche topics also revealed decisions about payment handling, caching and adoption of specific technology tools and platforms. However, as adoption of ADRs remains limited, the study highlighted gaps in design topics related to object oriented design, data-centric app architecture and patterns such as the Layered approach and security related patterns.
    \section{Sustainability Practices}
        When it comes to sustainability, ADRs appear to be a valuable tool that assists the capturing of architectural knowledge, avoiding knowledge vaporization, ensuring consistent and informed decision-making, and facilitating maintenance and evolution of software systems. ADRs, due to their ease of use and lightweight nature, can also help contribute to better impact analysis, understanding of design rationale, and knowledge sharing across teams and stakeholders, all while creating a traceable chain of decisions. Sustainable practices for ADR usage, derived from common complaints of software architects via a plethora of studies, include maintaining a centralized repository for ADRs that acts as a source of truth, integrating ADRs into regular project workflows, and encouraging collaborative decision-making processes all while making these processes as frictionless and time consuming as possible. Organizations, drawing inspiration from experienced practitioners and companies in the industry, could also provide training and resources to promote the effective use of ADRs among their architects and developers. In general, ADRs is a straightforward way to transform the architectural knowledge acquired from the architecting process from implicit to explicit. They help convey the ``why'' of a decision, not the ``what'' and ``how'', in an efficient and easy-to-manage way. By embedding ADRs into the development lifecycle and using tools like version control, teams can ensure that ADRs remain current, accessible, and useful. 
    \section{Future Directions}
        Future research on ADRs can focus on several directions. Firstly, expanding the dataset to include ADRs from private and enterprise environments could provide further understanding of architectural decision-making practices across different contexts.
        
        Leveraging a larger dataset, machine learning models can be trained to assist architects when making architectural decisions, replacing outdated and inflexible tools. This can also be integrated with commonly used architectural knowledge management tools to produce, store, index and manage fully automated ADR knowledge bases across organizations. Autonomous agents can also be leveraged to monitor developer's code via pull requests to check whether it conforms with the architectural standards proposed in previous ADRs from the organizational knowledge base. This could reduce architectural drift and provide real time architectural guidelines in the development process. However, it is unclear how many ADRs will be needed to achieve this, as research on ADRs and artificial intelligence methods is limited. 
        
        ADRs could also be incorporated as a topic in evidence-based software engineering (EBSE) studies \cite{evidence_based_software_eng} to assess their impact on decision-making quantitative metrics. Finally, exploring the creation of dependency graphs from ADR back-links, that encompass the whole architectural decision chain, is also an option. This could potentially trace the status of a software project at any given point in time, preventing conflict, and ensuring composability across decisions and system components.