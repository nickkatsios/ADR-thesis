\chapter{Research Methodology}
    \section{Research Approach}
        % General Overview of what was done
        In this study, a mixed-methods approach was adopted to comprehensively analyze the topics addressed in ADRs from a dataset gathered from GitHub by real practitioners. The goal was to uncover patterns and prominent architectural interests within the practitioner community and how those interests relate to the general topics of software architecture. 

        % General overview of how it was done and why it differs from others 
        Specifically, a dataset from a previous mining software repositories (MSR) study on ADRs, that resulted from a systematic process and guidelines, \cite{Github_study_ADRs, MSR_systematic_process} was leveraged to perform analysis on the records' contents using machine learning techniques and empirical observations to interpret and classify the results. While the researchers focused on finding out how widespread the use of ADRs has become, focusing on the trends and metadata in regards to the ADRs that were mined, this study's objective is to uncover the topics of the aforementioned records by getting insights into their contents and architectural fields of interest.

        % Dataset origin and why use it (skipped my modifications of the dataset since they are mentioned in the next chapter)
        The dataset was selected and deemed appropriate for the study for numerous reasons. Firstly, because of the recency of the data, dating back to 2020, four years before the present study was conducted. Secondly, because of the platform used to derive the dataset, as GitHub is the largest platform for open source software repositories by a significant margin\footnote{https://osssoftware.org/blog/open-source-software-repository-management/}. Lastly, because of the dataset quality and quantity, as it derived from an exhaustive automated search of all publicly available repositories  (estimated 128,411,417 as of February 26 2020\footnote{https://github.com/bugout-dev/mirror/blob/master/notebooks/rise-of-github.ipynb}) and the selection process was strict, involving manual and automated screening of results, applying concrete inclusion and exclusion criteria to identify repositories actually using ADRs for documenting ADDs, while excluding those with low quality documents.

        % Methods of analysis and why i used them
        For the identification of prevalent topics within the records' contents, a number of techniques were used, focusing on machine learning approaches applied using automated methods in Python. In detail, topic modeling, a text mining technique, was utilized for its effectiveness to extract abstract topics and word clusters from a collection of documents. This method was employed as it has also demonstrated efficacy in other software engineering empirical studies, analyzing text mostly related to developer communication and bug reports.\cite{Topic_modeling_in_software_engineering_research}. Approaches using Latent Dirichlet Allocation (LDA) were also employed as they constitute common practice in the topic modelling space, with mixed results. Combining all the findings, a specific stack of algorithms and techniques was chosen, described in detail in the following chapters and facilitated by the use of the BERTopic Python library. \cite{bertTopic}

        % Result interpretation
        To interpret the analysis results, multiple approaches were utilized as there is no definitive method to objectively label topic clusters, which also depends on the reader's perspective \cite{datasciencecentralTopicModeling}. In the absence of ground truth, keywords from the record clusters were analyzed and interpreted using large language models (LLMs) as representation models. The labels were then cross-checked with the most representative documents of each cluster, specifically those that are semantically closest to the center of the cluster based on their distance in the vector space. Additionally, in certain cases, manual topic naming was performed based on frequent words within a topic, a common practice in topic modeling to reduce ambiguity \cite{Topic_modeling_in_software_engineering_research}.
        
    \section{Research objectives and questions}
        As mentioned, The main objective of this study is to uncover prominent software architecture topics by examining a collection of ADRs. To achieve those research objectives,the following research questions were defined (RQs):

        \begin{enumerate}
            \item \textbf{RQ1}: What are the most frequently discussed software architecture topics in ADRs? Can these be sorted in a tree-like structure? 
            \item \textbf{RQ2}: How many topics from the general software architecture space are present in the ADRs? How many of them are missing?
            \item \textbf{RQ3}: Are there any outlier topics in software architecture? What are they about and how can they be sorted?  
        \end{enumerate}. 

        The aim of RQ1 is to outline the most prevalent topics that concern open source maintainers and owners of the repositories whose ADRs are being analyzed, effectively examining the most common design decisions projects face.
        
        RQ2 aims to comprehensively compare the identified topics with various aspects of software architecture to determine if they are aligned. This comparison could also provide insights into less prevalent software architecture topics as well as the breadth of the ADRs in the dataset and suggest future directions for enriching it.
        
        Finally RQ3, aims to discover the intricacies that outlier ADRs might have. This, on the one hand may help discover new architectural concerns or interesting architecture sub-domains that have not been yet thoroughly researched. On the other hand, it could also just discover ADRs that talk about topics too niche to be considered validating potential suspicions about their irrelevance.

    % Should this be here c
    \section{Analytical Methods}
        Analytical methods employed.