\chapter{Architectural Decision Making and Architectural Knowledge}
    \section{Architectural Decision Making}
        In their pursuit to meet functional and non functional requirements, software architects implicitly or explicitly make design decisions about the software architecture of a system. While research efforts in the past focused mainly on the view of software architecture as components and their connectors \cite{software_arch_in_practice_book}, the paradigm has shifted towards recognising those design decisions as first class entities.\cite{first-class-Arch-decisions}. This approach is considered to provide a better overview of the system's properties, as all decisions have profound impact on the system itself and its evolution can be traced more easily by serializing them into a "decision chain".

        The process of making an architectural decision, typically involves solving a problem by satisfying specific requirements, in a given context, by considering various alternatives and weighing their pros and cons to navigate through the solution space and finally reach a consensus on how to proceed on the implementation. A decision rationale can also be provided to justify the end result. These decisions can be made solely by a singe architect, a group of architects or, in more social approaches, a team of engineers and developers. The decision can determine a specific technology to be used \cite{developer-study-arch-decisions}or set standards for the whole system such as architectural patterns. For example, a technology-specific architectural decision may refer to the type of relational database to be used in a software project. The alternatives may include a MySQL database and a Microsoft SQL Server database. In the end maybe the MySQL database is used due to its open-source nature and lack of licensing requirements. A pattern decision may be related to the use of the model, view, controller (MVC) pattern in a web application. This defines the core of the system architecture and all subsequent decisions will depend on that view. 

        When finalizing architectural decisions, architects often omit considering certain aspects of the problem. First, rarely are architectural decisions standalone entities, and if one is solidified, it will most likely affect previous decisions and dependencies, making some of them obsolete. In addition, since the sum of decisions comprise the final architecture, architects often fail to consider how a decision can be reversed in the event of integration failure. Finally, the knowledge gained from considering different decisions is not a part of the system unless it has been documented. [cite here]

    \section{Architectural Knowledge}
        Architectural Knowledge (AK) is a multidisciplinary topic that is difficult to define. 
        An understanding from the reviewed literature \cite{architectural_knowledge_definitions} indicates that architectural knowledge includes not only the architectural design itself but also the design decisions, assumptions, context, and other factors that collectively explain why a particular solution is the way it is. This is in line with the decision making process discussed in the previous section. This knowledge spans the problem domain (defining what issues need to be addressed), through decision-making processes (how issues are to be addressed), to the solution domain (what has been implemented to address these issues) and is built up from many past collective decisions that are context specific (related to a specific system in question), also known as \textit{Application-specific knowledge} or domain specific (related to software architecture in the broad sense) also known as\textit{ Application-generic knowledge} \cite{Patterns+ArchDecisions}. This is why architects, often rely on intuition to derive the solution to an architectural problem. \cite{Patterns+ArchDecisions, archtitect_survey}.

        It is therefore evident that AK plays an important role in the whole software development life-cycle. It helps in laying a framework that defines how software components interact and integrate, which is crucial for ensuring system scalability, performance, and reliability. It encompasses the system's evolution so it can later be presented to all relevant stakeholders, and accurately calculate the cost of development and maintenance. It is also clear that AK should be widely shared rather than confined to silos or the expertise of specific architects. Given that AK consists of individual design decisions, these can be reusable and adaptable to various scenarios, thus saving both cost and effort. This cross-entity sharing however is only possible if AK is properly documented and managed.
        
    \section{Problems of software architecture and Architectural knowledge management}
    - see arch as design decicions
    - ARCHITECTURAL KNOWLEDGE SHARING (AKS)APPROACHES: A SURVEY RESEARCH
    \section{Mental models and Decision making}
        Role of experience and structured approaches.
    \section{Importance of Capturing AK}
        Importance of capturing AK for continuity and governance.
    \section{Overview of Tools, Evolution and Capabilities}
        Overview of traditional and modern tools for capturing AK.
        Evolution, capabilities, and comparison of these tools.