\chapter{Introduction}
    Software architecture has become an increasingly complex and multidimensional topic in the past years that gathers the interest and inspiration from various fields of study. When designing systems, various design decisions are made by architects that capture the functionality of the ever-increasingly intertwined components. By recognizing design decisions as first-class entities that address functional and non-functional requirements, software architecture is divided into many sub-tasks with logical continuity that may or may not depend on each other \cite{Arch+DesignDescisions}. This process, produces system knowledge also known as Architectural Knowledge (AK).
    However, without proper documentation, this knowledge often evaporates increasing the cost of maintenance and creating uncertainty and other adverse effects for all relevant system stakeholders. To combat this problem and provide information retention,  Architectural Design Records (ADRs) and relevant artifacts are employed as a strategic method to systematically document crucial decisions, capturing not only the outcome but also the rationale behind each decision. In the following chapters, a general overview of AK will be presented followed by ways to capture it using industry-validated tools and formats. Then, leveraging an enriched ADR dataset derived from a systematic mining software repositories (MSR) study, a set of questions will be proposed and answered to shed light on the trends and practices regarding architectural management using a plethora of analysis methodologies. Since ADRs are a highly practical topic, we will also observe how they are being used by practitioners to facilitate knowledge management via various mechanisms.

    % Overview of sections and why they were chosen
    The remainder of the study is structured as follows. Section 2 presents background concepts about architectural decision making and architectural knowledge management, providing context on the process of software architecture based on a literature review. Section 3 goes into detail about Architectural Decision Records (ADRs), including their definition, role, usage, types, tool support, and practical applications with the aim to comprehend how they can fit in the architecture life cycle and the benefits they provide. Section 4 explains the research methodology employed to propose research questions, gather and analyze the dataset of ADRs to uncover topics and trends. Section 5 further describes the ADR dataset and extraction process, including data collection, cleaning enrichment and quality. Section 6 presents the ADR dataset analysis, explaining the analysis methods and discussing the results to the research questions. Finally, Section 7 addresses the conclusions derived from the analysis and proposes sustainability practices for ADRs along with future research directions.