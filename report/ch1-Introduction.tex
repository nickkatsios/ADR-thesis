\chapter{Introduction}
    Software architecture has become an increasingly complex and multidimensional topic in the past decade that gathers the interest and inspiration from various fields of study. When designing systems, even before the implementation, various design decisions are made by architects that capture the functionality of the ever-increasingly intertwined components. By recognizing design decisions as first-class entities that address functional and non-functional requirements, software architecture is divided into many sub-tasks with logical continuity that may or may not depend on each other.\cite{Arch+DesignDescisions} This process, produces invaluable knowledge also known as Architectural Knowledge (AK). However, without proper documentation and the required tooling, this knowledge often evaporates increasing the cost of maintenance and creating uncertainty and other adverse effects for all relevant system stakeholders. To combat this problem and provide information retention,  Architectural Design Records (ADRs) and other artifacts are employed as a strategic method to systematically document crucial decisions, capturing not only the outcome but also the rationale behind each decision. In the following chapters, a general overview of AK will be presented followed by numerous ways to capture it using industry-validated tools and formats. Then, leveraging an enriched ADR dataset from open-source repositories and real world companies, a set of questions will be proposed and answered to shed light on the trends and practices regarding architectural management using a plethora of analysis methodologies.
    - Do a summary of each section and its goal